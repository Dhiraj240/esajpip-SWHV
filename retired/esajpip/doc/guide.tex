\chapter{User guide}

\section{Introduction}
\label{introduction}

The ESA JPIP server is capable to handle the following types of JPEG2000 image files:
raw J2C, JP2 and JPX with or without hyperlinks. The codestreams of the images
must comply with the following requirements:

\begin{itemize}
\item No tiles partition is allowed.
\item The progression order must be LRCP, RLCP or RPCL.
\item PLT markers must be included with the information of all the packets.
\end{itemize}

If one image does not fit these requirements then it can not be served by this
server. Moreover, although they are not mandatory, the following requirements
are strongly recommended:

\begin{itemize}
\item No tile-parts.
\item Use the RPCL progression order.
\item Use an appropriate precinct partition.
\end{itemize}

The ESA JPIP server does not transcode the images at all, serving them as it. 
Therefore the last requirement is particularly recommended for big images in
order to improve the transmission efficiency. For resolution levels bigger 
than $1024 \times 1024$ precincts of $512 \times 512$ or $256 \times 256$ are
recommended.

\section{Installation}

In order to install the server it is necessary to get the code (for example,
from \href{https://code.launchpad.net/esajpip}{Launchpad}) and compile it.
The code of the server has been specifically written for Linux systems, so it
has not been tested in any other platform.

For compiling the source code the libraries libproc, log4cpp and libconfig++
must be installed, for developing, on the system (see Section \ref{libraries} 
for more details about these used libraries). 

Just by means of one ``make'' the compilation is performed. If ``make doc'' is
used, this documentation is created automatically mixing this text with the
source code documentation (\href{http://www.stack.nl/~dimitri/doxygen/}{Doxygen} 
and LaTeX are required to be installed).

In order to get running the application it is only necessary to modify the
configuration file as needed (see Section \ref{config}) and to have a caching
directory with the write permission enabled for the server.

\section{Configuration}
\label{config}

The configuration of the different parameters of the server is carried out by
means of the file ``server.cfg'', which must be located in the same directory
where the server is launched. The structure of this file is:

\begin{itemize}
	\item Section ``\textit{listen\_at}'':
		\begin{itemize}
			\item Field ``\textit{port}'': Port used for listening.
			\item Field ``\textit{address}'': IP server address used for listening. This
field can be empty meaning that the server will listen at any address available.
		\end{itemize}

	\item Section ``\textit{folders}'':
		\begin{itemize}
			\item Field ``\textit{images}'': Root of the folder where the images to serve
are stored.
			\item	Field ``\textit{caching}'': Root of the folder to store the cache files.
			\item	Field ``\textit{logging}'': Folder to store the log files.
		\end{itemize}
	\item Section ``\textit{connections}'':
		\begin{itemize}
			\item Field ``\textit{time\_out}'': It defines the timeout (in seconds) of every connection.
			\item	Field ``\textit{max\_number}'': The maximum number of simultaneous connections.
		\end{itemize}
	\item Section ``\textit{general}'':
		\begin{itemize}
			\item Field ``\textit{logging}'': It indicates if the log file is created (1 - Yes, 0 - No).
			\item	Field ``\textit{cache\_max\_time}'': Expiration time of the cache files, in seconds. If
this value is less than zero it means that no expiration time is used for the cache files.
			\item	Field ``\textit{max\_chunk\_size}'': Maximum chunk size used for transmission, in bytes.
		\end{itemize}
\end{itemize}

Each time the server opens an image to be served, it creates a little associated cache file,
if it does not exist yet, with the related indexing information, within the configured
caching folder.

Here is an example of a configuration file, which is the default one
included in the Launchpad repository:

\lstinputlisting[numbers=none,xleftmargin=50pt]{../../server.cfg}


\section{Commands}
\label{commands}

The application accepts the following command line parameters:

\begin{itemize}
	\item \textit{esa\_jpip\_server [start]}: It runs the server application.
	\item \textit{esa\_jpip\_server status}: It shows some information (memory, num. of connections, etc.) of the current server process.
		Currently the information shown is:
		\begin{itemize}
			\item The available total memory.
			\item	The memory consumed by the father process.
			\item	The memory consumed by the child process.
			\item	The number of connections.
			\item	The number of iterations (the number of times that the child process has been restarted).
			\item	The number of threads of the child process.
			\item	The CPU usage of the child process.
		\end{itemize}
	\item \textit{esa\_jpip\_server record [name\_file]}: It shows the same information in columns, being updated every 5 seconds. It accepts a third parameter, a name of a file where to store this information.
	\item \textit{esa\_jpip\_server stop [child]}: Both processes or only the child process (depending on the second parameter) associated to the current server running are finished. 
	\item \textit{esa\_jpip\_server debug [child]}: It calls the debugger for the parent or child process depending on the second parameter.
 	\item \textit{esa\_jpip\_server clean cache}: It removes all the cache files from the cache root folder which have exceeded the ``cache\_max\_time'' field from the ``server.cfg'' file.
It also removes the existing ``.backup'' files from the same directory.
\end{itemize}


\chapter{Developer guide}

\section{Introduction}

The idea for the development of this application has been to implement a very
stable and scalable solution of a JPIP server, specifically designed for Linux
systems in order to fully profit from its characteristics. No portability
strategies has been followed.

The language chosen for the development is C++, writing the code following  as
much as possible the
\href{http://google-styleguide.googlecode.com/svn/trunk/cppguide.xml}{Google
style guide}. The code is commented according to the Javadoc format, what
allows to use the Doxygen tool and generate the documentation of the
Part \ref{sourcecode} automatically. This documentation includes, if the tool
\href{http://www.graphviz.org/}{Graphviz} is installed, the collaboration diagrams 
of all the classes, as well as the call diagrams of all the functions are 
generated.

No libraries strictly related to the server performance have been used, except
from the \href{http://www.sgi.com/tech/stl/}{standard STL library} and 
\href{https://computing.llnl.gov/tutorials/pthreads/}{pthread}. 
The used libraries (see Section \ref{libraries}) are only related to secondary 
features of the application. 

For the data/file
operations the classes of the namespace \hyperlink{namespacedata}{data} 
are implemented. For example, 
the class \hyperlink{classdata_1_1BaseFile}{data::BaseFile} is a 
safe wrapper for the
FILE related functions; the classes \hyperlink{classdata_1_1InputStream}
{data::InputStream} and \hyperlink{classdata_1_1OutputStream}
{data::OutputStream} allow the binary serialization of the classes;
etc. 

Regarding the IPC and threads management operations the classes of the
namespace \hyperlink{namespaceipc}{ipc} have been designed. For example,
the class \hyperlink{classipc_1_1Mutex}{ipc::Mutex} implements the
logic of a mutex; the class \hyperlink{classipc_1_1RdWrLock}
{ipc::RdWrLock} implements a reader/writer lock based on the 
pthread\_rwlock functions; etc.

Working with socket has been easy thanks to the classes of the namespace
\hyperlink{namespacenet}{net}. For example, the class 
\hyperlink{classnet_1_1Socket}{net::Socket} offers a powerful interface
for the socket Linux functions; the class \hyperlink{classnet_1_1SocketStream}
{net::SocketStream} allows working with sockets as continuous data streams,
compatible with the stream classes defined in the STL library; etc.

Finally, for working with the HTTP protocol the classes of the namespace
\hyperlink{namespacehttp}{http} have been implemented. Among others, the
classes \hyperlink{classhttp_1_1Request}{http::Request} and 
\hyperlink{classhttp_1_1Response}{http::Response} permit to generate/generate
HTTP messages, being compatible with the STL streams.

With the help of the library \href{http://log4cpp.sourceforge.net/}{log4cpp}
a trace system has been designed in order to easy the server logging as well
as the debugging. The macros ``LOG'' and ``ERROR'' are defined to log
information and error messages respectively, whilst the macro ``TRACE'' is
defined to put debugging information in any part of the code. The debugging
logs can be enabled/disabled by means of the macro ``SHOW\_TRACES''. For
information see the content of the file \hyperlink{trace_8h}{trace.h}.

Before explaining in detail the server architecture in Section \ref{sec:architecture},
some concepts of the JPEG2000 standard (Section \ref{sec:jpeg2000}), as well as
the JPIP protocol (Section \ref{sec:jpip}) are explained, necessary to
correctly understand the server code.

The classes of the server code related to the JPEG2000 standard are defined in
the namespace \hyperlink{namespacejpeg2000}{jpeg2000}. The classes for working
with the JPIP protocol are located within the namespace \hyperlink{namespacejpip}
{jpip}. Within the text of the following sections, references to some classes
of these namespaces are given.

\section{The JPEG2000 standard}
\label{sec:jpeg2000}
Part 1 of the \href{http://www.jpeg.org/jpeg2000/}{JPEG2000} standard describes 
a core compression system that is based on the dyadic 
\href{http://en.wikipedia.org/wiki/Discrete_wavelet_transform}
{DWT (Discrete Wavelet Transform)}
and the 
EBCOT (Embedded Block Coding with Optimal Truncation). Some features of this 
compression system are high compression ratios, error-resilience, lossless 
and lossy compression, random access to the compressed stream, resolution and 
quality scalability, and support for multiple components. 
These characteristics make it ideal for the coding and retrieving of 
large remote images.

\subsection{Data partitions}

The JPEG2000 standard defines a wide variety of partitions
for the image data, with the aim of exploiting at the maximum the
offered scalability. All of these partitions
allow an efficient manipulation of the image, or a part of it. Fig. 
\ref{fig:partitions} shows a graphical example of the main partitions.

\begin{figure}[!b]
  \begin{center}
    \resizebox{0.95\textwidth}{!}{\input{../partition}}
  \end{center}
  \caption{Data partition defined by the JPEG2000 standard.}
  \label{fig:partitions}
\end{figure}

In order to understand the concept of each partition defined
in the JPEG2000 standard, it is necessary to clarify the concept
of canvas. The canvas is a bidimensional drawing zone where
all the partitions are mapped to form the related image.
Hereinafter, all the used coordinates are in relation to 
a canvas, which size, width ($I_{2}$) and height ($I_{1}$), corresponds to
the total size of the associated image. Each partition is 
located and mapped over the canvas in a specific way.
An image is composed by one or more components. In the most
of the cases the images have only three components: red, green
and blue (RGB), with a size equals to the canvas.

The JPEG2000 standard allows to divide an image into smaller 
rectangular regions called tiles. Each tile is compressed 
independently in relation to the rest, hence the compression
parameters can be different among them. By default there is
always one tile as minimum, which equals to the whole image.

One of the possible applications of the tile partitioning is its use
with images that contain different elements and visually separated,
like text, graphics or photographic materials. When this does not
occur, and the images are continuous and homogeneous, the tiling
is not recommended because it produces artifacts in the borders of
the tiles, causing a mosaic effect. Moreover, the size of a
compressed image is larger when the tiles are used.

The DWT transform and all the quantification/coding stages
are applied independently to each tile-component. A tile-component,
of a tile $t$ and a component $c$, is defined by a bidimensional
zone limited by $t$ taking into account the zone occupied by
$c$. This means that, if an
image has only one tile, with three color components, there are
three tile-components, which are compressed independently.

For each tile-component, identified by the tile $t$ and the component
$c$, there are a total of $D_{t,c} + 1$
resolutions, where $D_{t,c}$ is the number
of DWT stages applied. The $r$-nth resolution level of a compressed
tile-component is obtained after applying $r$ times the inverse DWT
transform. The
$r$ value is in the range of $0 \leq r \leq D_{t,c}$. 

Each tile-component, after being applied the DWT, is 
divided into code-blocks, that are coded independently. 
In each resolution $r$ of each tile-component $(t,c)$, the
code-blocks are grouped in precincts. This partition is
defined by the height, $P_{1}^{t,c,r}$,
and the width, $P_{2}^{t,c,r}$,
of each precinct. The number of precincts in vertical, 
$N_{1}^{P,t,c,r}$, 
as well as in horizontal, $N_{2}^{P,t,c,r}$ are given by
the following expression:

\begin{equation*}
N_{i}^{P,t,c,r} = \left\lceil \frac{I_{i}}{2^{D_{t,c}-r}P_{i}^{t,c,r}} \right\rceil
\end{equation*}

Code-blocks refer to the wavelet coefficients generated by the DWT
transform, thus rectangular regions within the wavelet
domain. However, precincts refer to rectangular regions within the
image domain. The spatial scalability offered by the standard is
carried out with the precincts.  

The packet is the fundamental unit for the organization
of the compressed bit-stream of an image. Each precinct contributes
to the bit-stream as many packets as quality layers there are. The
compressed data of each code-block is divided in different segments
called quality layers. All the code-blocks of all the precincts
of the same tile are divided into the same number of quality layers,
although the length of the quality layers between code-blocks can be
different (the length can be even zero). For a certain layer $l$,
the set of all the layer $l$ of all the code-blocks related to a
precinct form a packet.

In order to decode a certain region of an image it is necessary
decode all the packets related to that region. In the server code,
the class \hyperlink{classjpip_1_1WOIComposer}{jpip::WOIComposer} allows to know,
for a given region of interest, hereinafter called WOI (Window
Of Interest), all the required packets to decode it. 

A packet $\zeta_{t,c,r,p,l}$ 
is identified by the tile $t$, the
component $c$, the resolution $r$, the precinct $p$ (in precinct
coordinates)
and the quality layer $l$. In the server code, the class
\hyperlink{classjpeg2000_1_1Packet}{jpeg2000::Packet} is used to
identify a packet.

\subsection{Code-stream organization}

Part 1 of the JPEG2000 standard defines a basic structure for
organizing the image compressed data into code-streams. A code-stream
includes all the packets generated by a compression process of an image
plus a set of markers, that are used for signaling certain parts, as
well as for including information necessary for the decompression. 

The code-stream is itself a simple file format for JPEG2000 image.
Any standard decompressor must be able to understand a code-stream
stored within a file. This basic format is also called raw, and
its most used extension is ``.J2C''.

The markers have an unique identifier, that consists of an unsigned
integer of $16$ bits. These markers can be found alone, that is,
only the identifier, or accompanied by additional information,
receiving in this case the name of marker segment.

The marker segment has, after the identifier, another unsigned
integer of $16$ bits with the length of the included data, including
as well the two bytes of this integer, but without counting the
two bytes of the identifier.

The code-stream always begins with the SOC (Start Of Code-stream)
marker, which does not include any additional information. 
After this marker a set of markers called ``main header'' begins.
After the SOC marker there is always a SIZ marker, with global 
information necessary for decompressing the data, e.g. the image
size, the tile size, the anchor point of the tiles, the number
of components, the sub-sampling factors, etc.

There are another two markers that are mandatory in the main header:
COD, with information related to the coding of the image, like the
number of layers, number of DWT stages, the size of the code-blocks,
the progression, etc.; and QCD, which contains the quantization
parameters. These two markers can be stored in any position within the
main header.

The rest of the code-stream, until the EOC (End Of Code-stream),
located just at the end of it, is organized as it is shown
in Fig. \ref{fig:code-stream}. For each image tile, there is
a set of data. This data is divided into one or more tile-parts.
Each tile-part is composed by a header and a set of packets.
The header of the first tile-part is the main header of the tile.
The header of each tile-part begins with the SOT (Start Of Tile)
marker and ends with the SOD (Start Of Data) marker, starting then
the related sequence of packets, according to the last COD or POC
marker. The main header ends when the first SOT is found.

\begin{figure}[!t]
  \begin{center}
    \resizebox{0.65\textwidth}{!}{\input{../codestream}}
  \end{center}
  \caption{Code-stream organization.}
  \label{fig:code-stream}
\end{figure}

In order to permit a random access to the data of a code-stream,
that by default is not feasible, JPEG2000 offers the possibility
of including the TLM, PLM and/or PLT markers. The TLM and PLM markers
are included within the main header, whilst the PLT marker goes
in the header of a tile or tile-part. The goal of the TLM marker
is to store the length of each tile-part that appear within the
code-stream. This length includes the header as well
as the set of packets, so for knowing where is the beginning of the
data it is necessary to analyze firstly the header. The PLM marker
stores the length of each packet of each tile-part of the code-stream.
Each packet of the code-stream has a certain length, that can
not be known a priori. Therefore including this marker facilitates
a random access of the packets. The PLT marker has the same function
as the PLM marker, but at the level of tile-part, thus it stores
the length of all the packets of the belonging tile-part. This
marker is commonly most used than PLM. 

The PLM and PLT markers produces an increase of the code-stream
length, although the way of coding the packet lengths helps to
avoid an excessive overhead: a length $L$ of a certain packet,
that can be represented with $B_{L}$ bits, is stored coded
with $\left \lceil \frac{B_{L}}{7} \right \rceil$ bytes. For a
length $L$ is generated a sequence of bytes where only the
less significant $7$ bits are used. The most significant
bit of each byte indicates if the belonging byte is ($1$) or
not ($0$) the last one of the sequence. This way of numeric
encoding is widely used in Part 9 of the standard, specially with
the JPIP protocol. With this protocol, to each variable sequence
of bytes that represents a number encoded in this way is called
VBAS (Variable Byte-Aligned Segment). The class
\hyperlink{classjpip_1_1DataBinWriter}{jpip::DataBinWriter}, within the server code,
contains methods to generate VBAS coded values.

\subsection{Progressions}
\label{sec:progresiones}

The packets generated by the JPEG2000 compression process
are neither independent nor self-contained. Having a certain packet,
it is not possible to figure out to which part of the related image
it belongs without additional information. The length of the packet
can not be determined before being decoded, and many packets can not
be decoded without decoding other packets before. This is why it is
necessary to include markers like TLM, PLT or PLM, previously
commented, in order to allow a random access without decoding.

The packets of each tile-part appear according the progression
specified by the last COD or POC marker read, before the
SOD marker. Part 1 of the JPEG2000 standard defines 5 possible kinds of
progressions for ordering the packets within a tile or tile-part.
Each progression is identified by means of a combination
of four letters: ``L'' for quality layer, ``R'' for resolution
level, ``C'' for component and ``P'' for precinct. Each letter identifies
the partition of the progression. Hence for the LRCP progression,
for example, the packets would be included as follows:\\
\\for each layer $l$\\
\hspace*{1cm} for each resolution $r$\\
\hspace*{2cm} for each component $c$\\
\hspace*{3cm} for each precinct $p$\\
\hspace*{4cm} include the packet $\zeta_{t,c,r,p,l}$\\

The different progressions allowed by the standard are: LRCP, RLCP,
RPCL, PCRL and CPRL. To choose a progression or another depends on 
the application to
develop, and how the packet must to be decoded. For example,
if the packets are going to be accessed randomly, but 
as minimum disk accesses as possible are required, RPCL would
be the ideal progression in this case. 
In the case of image transmission, the packets must also follow a
specific order or progression when they are transmitted. 
When an image is transmitted from a server to a client, the most
desired goal is to allow the client to be able to show reconstructions
of the image with a quality that is increased as the data is received.
The quality of the reconstruction must be always the maximum possible
according to the received data. Under this criteria, the LRCP progression
can be confirmed as the best one, and it is the progression used by
the class \hyperlink{classjpip_1_1WOIComposer}{jpip::WOIComposer}.

\subsection{File formats}

Although the code-stream is completely functional as a basic
file format, it does not allow to include additional information
that could be necessary in certain applications, e.g. meta-data,
copyright information, or color palettes. By means of the COM
marker auxiliary information can be included within a 
code-stream, but it is not classified nor organized in
a standard way.

Part 1 of the standard also defines a file format based on ``boxes'' that
allows to include, for example, in the same file, several code-streams
and diverse information correctly identified. These files usually have
the extension ``.JP2'', extension also used for calling this kind
of files.

The JP2 files are easily extensible. A basic structure of box is defined,
which can contains any kind of information. Each box is unequivocally 
classified by means of a $4$-bytes identifier. A file can contain
several boxes with the same identifier. The standard proposes an
initial set of boxes, that may be extended  according to specific
requirements. In fact, the JP2 format is the base of the rest of
formats and extensions defined in the rest of parts of the standard.

Each box has a header of $8$ bytes. The first $4$ bytes, $L$, form
an unsigned integer with the length in bytes of the content of the
next $4$ bytes, $T$, contain the identifier of the kind of box. This
identifier is commonly treated like a string of $4$ ASCII characters.
The value of $L$ includes the header, hence the real length of the
content of the box is $L - 8$. $L$ can have any value bigger or
equal to $8$, but also $1$ or $0$. If $L = 1$ the length of the
content of the box is coded as an unsigned integer of $8$ bytes, $X$,
located after $T$. In this case the header occupies $16$ bytes and
the length of the content is then $X - 16$. If $L = 0$ the
length of the box content is undefined, being possible only for the
last box of the image file.

Boxes can contain another boxes inside. It is possible to know
whether a box contains or not sub-boxes depending on the value of
$T$. If a box contains sub-boxes, it only can contain sub-boxes, 
so it can not combine sub-boxes with other data.

Within the server code, the class \hyperlink{classjpeg2000_1_1FileManager}
{jpeg2000::FileManager} contains all the necessary code to read and parse
JPEG2000 image files, from simple raw J2C files to complex JPX ones with
hyperlinks. When this class parses an image file, extract the associated
index information and stores it in an object of the class
\hyperlink{classjpeg2000_1_1ImageInfo}{jpeg2000::ImageInfo}.
           


\section{The JPIP protocol}
\label{sec:jpip}

Part 9 of the JPEG2000 standard is almost
completely dedicated to the development of systems for
remote browsing of images. It defines a set of technologies
(protocols, file formats, architectures, etc.) that allow
to exploit efficiently all the characteristics of the JPEG2000
compression system for the development of this
kind of systems. The technology which Part 9 mainly focuses on is 
the JPIP protocol (JPeg2000 Interactive Protocol). 

\subsection{Architecture}

The common client/server architecture of a system for remote browsing of
images that uses the JPIP transmission protocol has the organization
shown in Fig. \ref{fig:jpip}.

\begin{figure}
\resizebox{\textwidth}{!}{
\input{../jpip_fig}
}
\caption{Client/server architecture of the JPIP protocol.}
\label{fig:jpip}
\end{figure}

As it can be observed in the figure, the client is divided
into four functional modules. The Browser module consists of
the interface which the user interacts with in order to
define the desired WOI. JPIP allows to use multiple parameters
to define the WOI of the remote image to show in form of 
restrictions related to the partitions to transmit (number 
of quality layers, components, etc.). Geometrically,
JPIP always imposes that the WOIs have to be defined in
rectangular regions over a certain resolution level.

Once the user has defined the desired WOI, it is sent to the
``JPIP client'' and ``Decompressor'' modules. The first one is
the responsible for communicating to the JPIP server 
in order to send it the WOI using the messages and syntax
defined by JPIP. When the server receives this information,
it extracts from the associated image the required data for
the reconstruction of the WOI, and send it to the client.
This data is sent encapsulated in data-bins. In JPIP the
different elements of the data partitions of a JPEG2000 image
are reordered and encapsulated in data-bins, being the
transmission units.

As the client receives the data-bins from the server, it stores them
in a internal cache (``Cache'' module). The client cache is organized 
in data-bins as well. This cache is continuously used by the ``Decompressor''
module for generating progressive reconstructions of the WOI, 
that are passed to the ``Browser'' module to show to the user.

As it can be seen, except the ``Cache'' module which serves as a container,
the rest of the modules that composes the client work in parallel.
The Browser is continuously indicating to the JPIP client which
WOI is wanted by the user. The JPIP client keeps communicating to
the server for transmitting this WOI and, as it receives the data,
stores it within the cache. The decompressor generates every time
reconstructions of the current WOI with the existing data in the cache.
The modules begin their execution when the user defines the first WOI,
and they stop when all the data of the current WOI have been received.
The user can indicate a new WOI whenever, without
having to wait for the completion of the previous one.

The client cache stores all the data-bins received from the server, for all 
the WOIs of all the images explored by the user. The server 
maintains a model of the content of this cache, that allows it to avoid to
send to the client data that it already has. For instance, let's assume a user, after
exploring a certain $WOI_A$, requests then a new $WOI_B$ that has an
overlapping with $WOI_A$. If a server maintains updated the cache model of the
content of the client cache of the user, when the $WOI_B$ is requested, the
server only sends the data-bins necessary to reconstruct the part of $WOI_B$
that is not in common with $WOI_A$. If all the data-bins that allow to
construct a certain $WOI_i$ are defined by $D(WOI_i)$, we can say that the
server sends the data-bins $D(WOI_B) - \left( D(WOI_A) \bigcap D(WOI_B) \right)$.

The JPIP protocol was designed to be independent of
the used base protocol. However, the \href{http://www.ietf.org/rfc/rfc2616.txt}
{HTTP protocol}
is mainly used as base protocol, because
of the similarity of the syntax with JPIP. Moreover,
when the communication is based on HTTP, the \href{http://en.wikipedia.org/wiki/Chunked_transfer_encoding}
{chunked transfer encoding} is commonly used. The ESA JPIP server
always uses this kind of communication.

\subsection{Data-bin partition}
\label{sec:data-bins}

Data-bins encapsulate different items of the partition
of a JPEG2000 image. This new partition of data defined
by JPIP permits to identify and gather the different parts
of an image for their transmission. In the server code,
the class \hyperlink{classjpip_1_1DataBinWriter}{jpip::DataBinWriter}
is the responsible to generate the data-bins for the
transmission.

When a client/server communication is started, it is defined
which kind of data-bin stream is going to be used. JPIP
defines two kinds of streams, JPP, oriented to precincts, and
JPT, oriented to tiles. The stream type indicates what kind
of data-bins are used for the data transmission. The most used 
stream type is JPP and is the only one supported by the ESA 
JPIP server.

Each data-bin is unequivocally identified by the image or code-stream
to what belongs, the data-bin type and its identifier. This information
is included in the header of each data-bin.

The data-bins can be segmented for theirs transmission. A certain
set of data-bins can be divided into a random number of
segments, and they can be sent following any order. Each 
segment of data-bin includes the necessary information for
its correct identification.

In Table \ref{tab:data-bins} the relation of the
different types of data-bins defined in JPIP can be observed,
showing which stream type they belongs and what information
they contain. Next the main kinds of data-bins are explained:

\begin{table}
\begin{center}
\begin{tabular}[]{{l}{l}{l}}
\textbf{Data-bin} & \textbf{Stream} & \textbf{Contained information} \\
\hline \hline 
Precinct 		& JPP 		& All the packets of a precinct. \\\hline
Precinct extended 	& JPP 		& The same content with additional content. \\\hline
Tile 			& JPT 		& All the packets and markers of a tile.\\\hline
Tile extended 		& JPT 		& The same content with additional content. \\\hline
Tile header 		& JPT 		& The markers of the header of tile. \\\hline
Header	 		& JPP/JPT 	& The markers of the header of a code-stream. \\\hline
Meta-data 		& JPP/JPT 	& Meta-data of an image. \\
\hline \hline
\end{tabular}
\caption{List of the data-bins defined by JPIP.}
\label{tab:data-bins} 
\end{center}
\end{table}

\begin{itemize}

\item \textbf{Precinct data-bin:} Contains all the packets of
a precinct for a resolution $r$, a component $c$ and a tile $t$.
The packets are ordered within a data-bin for quality layer,
in increasing order.

Every precinct data-bin is identified by an index $I$ that is obtained
by means of the following expression:

\begin{equation*}
I = t + \left( (c + (s \times N_c)) \times N_t \right)
\end{equation*}

$N_c$ y $N_t$ are the number of components and the number of tiles
of the image to which the precinct belongs, respectively. The
index of tile, $t$, as well as the index of component $c$ and
the value $s$ begin with zero. To every precinct of every tile-component,
including all the resolutions, an unique number of
sequence $s$ is assigned. To all the precincts of the lowest resolution
a number of sequence is assigned that starts with zero to the one
located in the left-top border, and that continues increasing
by column and row respectively. In the next resolutions
a correlative sequence number is assigned to the precincts in the
same way.

\item \textbf{Tile data-bin:} Contains all the packets and markers associated
to a tile. It is composed by concatenating all the tile-parts
associated to a tile, including the markers SOT, SOD and all the
rest relevant markers. The data-bins of this kind are identified by means
of an incremental index, beginning with zero for the tile
located in the top-left border.

\item \textbf{Tile header data-bin:} Contains all the markers associated to 
a certain tile. It is formed by all the markers of the headers of all the
tile-parts of a tile, without including neither the SOT nor the SOD markers.
The identifier of this data-bin has the same numeration that in the case
of the tile.

\item \textbf{Header data-bin:} Contains the main header of the code-stream,
from the SOC (included) to the first SOT marker (not included). Neither the
SOD nor the EOC marker is included either.

\item \textbf{Meta-data bins:} These data-bins appear only if
the associated image is a file of the JP2 family. The meta data-bins contain 
a set of boxes of
the image file. The standard does not define
how these data-bins must be identified nor within which boxes 
must be stored, but when a meta data-bin is identified by zero,
means that it will include all the boxes contained by an image. 

\end{itemize}

\subsection{Sessions and channels}

A request of the client to the server can be either stateful or stateless.
The stateful requests are carried out within the context of a 
communication session, which state is maintained by the server.
The stateless requests do not require any session. The use of 
sessions improve the performance of the server because, for example,
when establishing a session with a certain image file, the server
opens that file and prepares it in order to be transmitted in
data-bins, so all the requests associated to that same session
do not provoke that the server makes again that process. The stateless
requests can be considered as unique sessions that end when the
server response ends.

Under a session, using stateful requests, the client can open
multiple channels, being able to perform multiple stateful
requests associated to the same session simultaneously. This
is specially useful for applications that show simultaneously
different regions of interest of the same image. The channels
can be closed and opened in independently, without affecting
to the session. To close a session would involve to close all
the channels opened associated to it.

A set of images is associated to each session. A session
implies that the server maintains a model of the client
cache. That model will be only maintained while
the session remains active. A channel, for a certain session,
is associated to a certain image and a kind of specific
data-bin stream (JPP or JPT). The channel is identified
in a unequivocal way by means of a alphanumeric assigned
by the server, and which format is completely free. The sessions
are not identified, since the identifier of the channel
must be enough to identify the channel as well as the session
to which belongs.

Many times clients are interested to maintain in the server
the client cache model between different sessions. The server
by default does not allow this possibility, beginning a new
session with a client always with a model of cache empty, although
the client cache contains initial information. In this case JPIP
defines a set of messages that allow the client to modify the model 
that the server of its cache. Therefore, when the client starts
a new session with a server, it would send a resume of the current
content of its cache to improve the communication and to avoid
redundancy, usually by means of a POST message.

The cache models maintained by the servers not only
allow to avoid the redundancy between messages, but they
are also used in order to allow the client to perform
incremental requests and control the stream of data. Clients can 
indicate in their requests a parameter with the maximum length
of the server response. Thanks to the cache model of
the server, the client can perform successive identical requests,
but varying that parameters, so the server sends as response
increments of information. Therefore the client has certain
control and can adapt the stream of information to the available
bandwidth and delay, but always taking into account that the server
can modify whenever the related parameter. This way of communication
is indeed the most common one in the existing implementations of
JPIP.

\subsection{Messages}

The JPIP requests are formed by means of a ASCII sequence of pairs
``parameter $=$ value''. This allows that a JPIP request can be encapsulated
within a GET message of the HTTP protocol, next to the 
character '?', concatenating all the pairs with the symbol '\&'. It is
hence formed a typical HTTP request when dynamic objects are referenced
like CGI.

Some of the available parameters for a JPIP request are the followings:

\begin{itemize}

\item \textbf{``fsiz=$R_{x}$,$R_{y}$'':} Specifies the resolution
associated to the required region of interest. The server chooses
the biggest resolution of the image so its dimension $R_{x}' \times R_{y}'$
satisfies that $R_{x}' \geq R_{x}$ and $R_{y}' \geq R_{y}$. In general
this parameter includes the resolution of the user screen.

\item \textbf{``roff=$P_{x}$,$P_{y}$'':} Specifies the position of the
upper left border of the required region of interest within the
indicated resolution. If this position is not indicated, the
server assumes the value $(0,0)$.

\item \textbf{``rsiz=$S_{x}$,$S_{y}$'':} Specifies the size
of the required region of interest. The server cuts this size
in order to fit it into the real image according to the
specified resolution.

\item \textbf{``len=number of bytes'':} Client tells with
this parameter to the server the maximum number of bytes
that it can include in the response. The server takes into
account this limit, not only in the response to the
current request, but in the rest of the next client responses
within the same session.

\item \textbf{``target=image'':} This parameter identifies the image
file from which to extract the specified region of interest.
When the HTTP protocol is used, this parameter is not necessary
since the name of the image is obtained from the own URL
specified in the GET message.

\item \textbf{``cnew=protocol'':} When the client wants to open
a new channel under the same session, it uses this parameter,
indicating the protocol that must be used for this new channel. The
types of valid base protocols are ``http'' and ``http-tcp''.

\item \textbf{``cid=channel identifier'':} When the client creates
a new channel, the server sends the channel identifier, that must
be included in all the requests associated to that channel.

\item \textbf{``cclose=channel identifier'':} The client may decide
to close certain channel by means of this parameter, just specifying
the identifier of that channel.

\item \textbf{``type=stream type'':} When a new channel is created, the
client indicates what kind of data-bin stream is wanted. The main
types of data-bin streams are JPP (``jpp-stream'') and JPT
(``jpt-stream'').

\item \textbf{``model=\ldots'':} How it has been commented previously,
the client can need to tell to the server the content of its cache
in order to update the content of the cache model maintained by
the last one. For example, \texttt{model=Hm,H*,M2,P0:20}
would tell the server to include in the cache model 
the main header of the code-stream, the headers of all the tiles,
the meta data-bin number 2 and the first $20$ bytes
of the precinct $0$.
\end{itemize}



\newpage

\section{Server architecture}
\label{sec:architecture}

\begin{figure}[!b]
\centering
\resizebox{0.8\textwidth}{!}{
\input{../architecture}
}
\caption{Server architecture.}
\label{fig:server}
\end{figure}

Fig. \ref{fig:server} shows a basic representation of the server architecture. It
consists of a hybrid model combining both process and thread approaches.

There are two processes, herein after called father and child. The first one
creates the second one and is who listens for new client connections. When
a new client connection is accepted by the father process, it sends this
connection to the child process through a UNIX-domain socket in order
to be attended. The child process is who provides all the functionality
to handle the client connections. Each client connection is handled
in the child process by a dedicated thread.

The father process maintains a vector with all the open connections. When
a connection is closed by the client, the child process notifies to the
father process, through the UNIX-domain socket, in order to remove the 
connection from the vector. 

Since the father does not perform relevant operations, its probability of
crashing due to an error is low. On the contrary, taking into account
that the child process is how maintains all the image indexes and attends
all the client connections, it may fail due to errors or not yet fixed
bugs. When a process is crashed, all the related threads and connections
are automatically closed. In the case of the child process, the current
open connections are not closed because they are shared with the father
process. This last can detect when the child process is down and launch
a new child process. This new process, considering that it inherits the
connections vector of the father process, it is capable to reestablish the
connections without affecting the clients, that is, the clients do not
notice when the child process crashes. 

This architecture provides a 
fault-tolerant and robust approach for the server, as well as it 
offers a good performance. The multi-threading solution
implemented in the child process is efficient in terms of memory 
consumption and fast sharing/locking mechanisms. Having separated
the client handling code from the father process provides 
robustness and security. In the source code of the main module
\hyperlink{esa__jpip__server_8cc}{esa\_jpip\_server.cc} can be
seen how this server architecture is created.

The configuration of the server is read from the file ``server.cfg''
(see Section \ref{config} for details). This configuration is parsed
by the class \hyperlink{classAppConfig}{AppConfig} and shared with both
the father and child process.

Between the father and child process is maintained a shared memory
block, carried out by the class \hyperlink{classAppInfo}{AppInfo}. This
allows to share information between the two processes, as well as to
provide a mechanism for getting information from other processes, like
in the case of the commands ``record'' and ``status'' (see Section
\ref{commands}).
 
The main parts of the child process, shown in Fig. \ref{fig:server}, are
the File manager, the Index manager and the Client manager, that
are explained in the following subsections.

Every client connection is handled by a Client manager module, in an
independent thread. This module manages the communication with the
client, serving as the JPIP interface with the sub-module Data-bin server. 
This sub-module contains the code to properly generate the data-bins
associated to the client requests, implementing as well the client
cache model.

The Index manager module is the responsible to manage the image indexes stored
in memory. When client requests to open a certain JPEG2000 image, the related
Client manager module requests the Index module for the associated 
indexing information. This module manages the indexing information in memory
of all the currently open images by all the Client manager modules.

The File manager module is in charge of parsing and extracting the indexing
information of the image files that are requested by the Index manager module.
The indexing information of each image is cached by means of little binary
``.cache'' files in order to avoid to repeat the indexing process when the
same image is open several times. These cache files are stored in the directory
specified in the configuration file.

\subsection{File manager}

The code of this module is contained in the class 
\hyperlink{classjpeg2000_1_1FileManager}{jpeg2000::FileManager}. Its main
function is to extract the indexing information of the image files.

The supported format files are J2C, JP2 and JPX, with or without 
hyperlinks. The files must also comply with the limitations specified 
in Section \ref{introduction}.

The indexing information of an image file is stored in an
instance of the class \hyperlink{classjpeg2000_1_1ImageInfo}
{jpeg2000::ImageInfo}. It
contains mainly a set of \hyperlink{classdata_1_1FileSegment}
{data::FileSegment} objects (pair of offset and length information
about a file segment) regarding the main headers, PLT segments,
metadata segments, etc., of the image.

The class \hyperlink{classjpeg2000_1_1ImageInfo}
{jpeg2000::ImageInfo} as well as all of its members variables
can be serialized using the serialization classes defined in
the namespace \hyperlink{namespacedata}{data}. This makes
easy load/save indexing information from the ``.cache'' files.

When the Index manager requests to the File manager the
indexing information of an image file, this last firstly checks
whether the associated ``.cache'' file exists or not. If it
exists, it just reads it, using the serialization to an 
\hyperlink{classjpeg2000_1_1ImageInfo}{jpeg2000::ImageInfo} object
and returns it. If there is not any cache file yet, it opens
the image file and parses it, generating the corresponding
cache file.

The cache files are stored in the directory defined in the configuration
file, and they are named using the full path of the image
files, replacing each directory separator '/' by underscores '\_',
and adding the extension ``.cache''. 

The server does not remove any cache file during its execution. They
can be removed either manually or by means of the server command
``clean cache'' using a LRU policy specified in the server configuration
file (see Section \ref{commands}).

\subsection{Index manager}

The code of this module is contained in the class 
\hyperlink{classjpeg2000_1_1IndexManager}{jpeg2000::IndexManager}. Its main
function is to manage the indexes of all the image files opened by the
server.

The index of each image is built from the indexing information retrieved
from the file manager module. Notice that the indexing information
generated by the file manager does not deal with packets, just the main
parts of the image, like the position of the main header or the
position of the PLT markers. From this initial information, the index
manager builds a complete index of all the packets of each image.

Each opened image is represented by an object of the class
\hyperlink{classjpeg2000_1_1ImageIndex}{jpeg2000::ImageIndex}. All
of these objects are handled by the index manager using a double linked
list in memory. 

When a J2C, JP2 or JPX (without hyperlinks) image file is opened, a
new \hyperlink{classjpeg2000_1_1ImageIndex}{jpeg2000::ImageIndex} is created,
if it does not already exist, and inserted in the list. When a JPX
image file with hyperlinks is opened, the index manager handles each
contained hyperlink independently. In this case a new node is also
created in the list, but it contains references to all the nodes
associated to its hyperlinks. Some of them may already exist in the list,
or they have been created.

Each image node in the list contains a reference count When this
count value gets zero, the node is removed from the list. The number
of references is incremented each time a client manager requests
for the associated image file, or when a JPX file is opened that
contains a link to that image file. 

When a client manager opens a new channel, it requests to the index 
manager for the index of the image associated to this new channel.
The index manager returns a reference of an object of the class
\hyperlink{classjpeg2000_1_1ImageIndex}{jpeg2000::ImageIndex}. 

The packet index (handled by means of the class 
\hyperlink{classjpeg2000_1_1PacketIndex}{jpeg2000::PacketIndex}) of
each image file is created on demand, by resolution level. In order
to support this feature, the image file must be compressed
with the progression order RPCL. On the contrary, the packet index
is fully created at the beginning, when the image file is opened.
This way of creating the index allows to adjust the memory
consumption required for the indexes according to the user
movements through the image.

The modifications (inserting or removing nodes) of the list of indexes 
are controlled by a mutex lock. However, each index node contains its
own independent locking mechanism for the modifications of the packet
index. Multiple threads, associated each one to a client manager,
can be working with the same image index node. If the packet index
does not allow to be generated on demand, the main operation of
all the threads over the packet index is to read. When the RPCL
progression is used and the packet index can be built on demand,
any thread can modify it (e.g. accessing to a new resolution
level). A reader/writer lock mechanism has been chosen for controlling
the modifications/accesses of the packet indexes, giving priority to
the read operations (the most common ones) over the write operations.

The class \hyperlink{classjpeg2000_1_1PacketIndex}{jpeg2000::PacketIndex}
uses internally the class \hyperlink{classdata_1_1vint__vector}
{data::vint\_vector} to stores the information of each packet. This
class allows to use vectors of integers where each integer value can
occupy a number of bytes not necessarily multiple of $2$. For example,
if an image contains less than $2^{24}$ packets, each packet can
be represented in the index by just $3$ bytes, instead of the minimal
standard $4$ bytes. 

The information stored of each packet in the index vector 
is only the associated file offset. If the image file contains several tile-parts, 
not recommended, thus dividing the set of packets in not contiguous segments, 
special markers are included to identify the offset jumps.


\subsection{Client manager}

The class \hyperlink{classClientManager}{ClientManager} contains the required
code to handle a client connection. When a new client connection is established
by the server, a new instance of this class is created passing a reference
to the application shared information (\hyperlink{classAppInfo}{AppInfo}), a
reference of the server configuration information (\hyperlink{classAppConfig}
{AppConfig}) and a reference to the Index manager (\hyperlink{classjpeg2000_1_1IndexManager}
{jpeg2000::IndexManager}). A new
thread is created for this connection which calls the method ``Run''
of the class.

The function of the code of the class \hyperlink{classClientManager}{ClientManager} 
is basically to parse the client requests (with the help of the class
\hyperlink{classjpip_1_1Request}{jpip::Request}) and form the appropriate responses,
all of them according to the JPIP protocol in HTTP format. It implements a basic
channel handling mechanism, as well as the interface with the Index manager, in
order to, when a new image file is requested by the client, get the associated
index.

Although the standard allows to use several channels over the same connection,
this implementation does not allow this feature. In practice, the most of the
applications do not use more than just one channel per connection, like for example
in the case of the viewer kdu\_show, from the Kakadu library, or JHelioviewer. This
means that the JPIP concepts of channel and session refer in this implementation
to the same thing. However, in the same connection, the client can open and
close several channels as many times as required, but considering that only
one channel can be opened at the same time.

In order to identify the channel opened by the client, a simple integer number
is used, which is incremented each time the client opens and closes a channel.
The target identifier returned by the server, when a channel is opened, and 
that is commonly used by the JPIP clients to perform local caching, is the
full path of the associated image file. This avoids the coherence problems
detected in other server implementations (like in the case of the kdu\_server)
when using complex hash values.

Taking into account that the child process can be restarted, and the client
manager might be started over an existing client connection, a mechanism to
find out the data already sent to the client, but without requesting it,
has been implemented. Each time all the complete data of a WOI has been
sent to the client, a little ``.backup'' file is created in the caching
directory with the content of the client cache model. Therefore, when
a client manager is started, it checks whether an associated ``.backup''
exists or not, loading its information as the current content of the
client cache model if so.

The main flow chart of the code of the client manager module can be observed in
Fig. \ref{fig:client_manager}. In the processing step of generating a new
chunk of data is when the data-bin server module is called. 

\begin{figure}
\centering
\resizebox{0.9\textwidth}{!}{
\input{../client_manager}
}
\caption{Flow chart of the client manager.}
\label{fig:client_manager}
\end{figure}


\subsection{Data-bin server}

The code of this module is located at the class \hyperlink{classjpip_1_1DataBinServer}
{jpip::DataBinServer}. This module manages the data that is served to a client, having
methods to generate chunks and data-bins, as well as it maintains the client cache
model.

The client cache model is managed with the help of the class 
\hyperlink{classjpip_1_1CacheModel}{jpip::CacheModel}. This class can be serialized,
what make easy saving/loading the ``.backup'' files generated by the client manager.

The data-bin server module uses the methods of the class 
\hyperlink{classjpip_1_1DataBinWriter}{jpip::DataBinWriter} for generating the
data-bins that will be sent to the client by the client manager.

Once the client manager receives a WOI request from the client, it is passed to
the data-bin server, which parses it and prepares the related resources for
the data-bin generation. For
instance, when the WOI specifies a set of hyperlinked layers within a JPX file, 
the data-bin server opens the associated files putting them ready for
extracting packets. An object of the class \hyperlink{classjpip_1_1WOIComposer}
{WOIComposer} is also prepared to explore the packets associated to the
WOI specified in the client request.

For each WOI request the client manager sends JPIP responses encapsulated in
HTTP messages, and using the chunked transfer encoding. The maximum length
of each chunk is determined by the value specified within the configuration
file. The method  ``GenerateChunk'' of the class \hyperlink{classjpip_1_1DataBinServer}
{jpip::DataBinServer} is able to fill a memory buffer, usually related to
a chunk data, with data-bins, according to the current client cache model and
the last passed request. This is the method used by the client manager to
generate the responses.

The data-bin server generates a new chunk of data-bins considering the data
already sent, recorded in the client cache model. Therefore, if the first
$N$ bytes of a precinct data-bin has been already sent to the client, in
the next chunk generation, these first $N$ bytes of the same precinct are
not included.

The sequence of data-bins that are included when a new data chunk is generated,
calling the ``GenerateChunk'' method, is as follows:

\begin{enumerate}

\item All the metadatas

\item For each codestream:
	\begin{enumerate}
	\item Main-header
	\item Tile-header
	\end{enumerate}

\item For each packet given by the \hyperlink{classjpip_1_1WOIComposer}{WOIComposer}:
	\begin{enumerate}
	\item For each codestream: Precinct
	\end{enumerate}

\end{enumerate}

Notice that, as it has been commented, the chunk generation and data-bin
inclusion is incremental according to the client cache model and the previous
sequence. For example, if
the metadata of an image file occupies $1$ MByte, and the chunk size is $1$ KB,
more than $1000$ chunks would be required to be generated before being able to
generate the first chunk with packet data (precinct). 

\section{Libraries}
\label{libraries}

The application uses the following libraries:

\begin{itemize}
	\item \textit{Libconfig++}: The libconfig++ library (\href{http://www.hyperrealm.com/libconfig/}{http://www.hyperrealm.com/libconfig/}) is used to read the
information of the configuration file of the server. The configuration 
file accepts C-like comments.
	\item \textit{Libproc}: The general information about the server process is shown thanks to this library (\href{http://packages.debian.org/sid/libproc-dev/}{http://packages.debian.org/sid/libproc-dev/}).
	\item \textit{Log4cpp}: This library is in charge of flexible logging to file (\href{http://log4cpp.sourceforge.net/}{http://log4cpp.sourceforge.net/}).
\end{itemize}



