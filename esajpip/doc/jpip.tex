
Part 9 of the JPEG2000 standard is almost
completely dedicated to the development of systems for
remote browsing of images. It defines a set of technologies
(protocols, file formats, architectures, etc.) that allow
to exploit efficiently all the characteristics of the JPEG2000
compression system for the development of this
kind of systems. The technology which Part 9 mainly focuses on is 
the JPIP protocol (JPeg2000 Interactive Protocol). 

\subsection{Architecture}

The common client/server architecture of a system for remote browsing of
images that uses the JPIP transmission protocol has the organization
shown in Fig. \ref{fig:jpip}.

\begin{figure}
\resizebox{\textwidth}{!}{
\input{../jpip_fig}
}
\caption{Client/server architecture of the JPIP protocol.}
\label{fig:jpip}
\end{figure}

As it can be observed in the figure, the client is divided
into four functional modules. The Browser module consists of
the interface which the user interacts with in order to
define the desired WOI. JPIP allows to use multiple parameters
to define the WOI of the remote image to show in form of 
restrictions related to the partitions to transmit (number 
of quality layers, components, etc.). Geometrically,
JPIP always imposes that the WOIs have to be defined in
rectangular regions over a certain resolution level.

Once the user has defined the desired WOI, it is sent to the
``JPIP client'' and ``Decompressor'' modules. The first one is
the responsible for communicating to the JPIP server 
in order to send it the WOI using the messages and syntax
defined by JPIP. When the server receives this information,
it extracts from the associated image the required data for
the reconstruction of the WOI, and send it to the client.
This data is sent encapsulated in data-bins. In JPIP the
different elements of the data partitions of a JPEG2000 image
are reordered and encapsulated in data-bins, being the
transmission units.

As the client receives the data-bins from the server, it stores them
in a internal cache (``Cache'' module). The client cache is organized 
in data-bins as well. This cache is continuously used by the ``Decompressor''
module for generating progressive reconstructions of the WOI, 
that are passed to the ``Browser'' module to show to the user.

As it can be seen, except the ``Cache'' module which serves as a container,
the rest of the modules that composes the client work in parallel.
The Browser is continuously indicating to the JPIP client which
WOI is wanted by the user. The JPIP client keeps communicating to
the server for transmitting this WOI and, as it receives the data,
stores it within the cache. The decompressor generates every time
reconstructions of the current WOI with the existing data in the cache.
The modules begin their execution when the user defines the first WOI,
and they stop when all the data of the current WOI have been received.
The user can indicate a new WOI whenever, without
having to wait for the completion of the previous one.

The client cache stores all the data-bins received from the server, for all 
the WOIs of all the images explored by the user. The server 
maintains a model of the content of this cache, that allows it to avoid to
send to the client data that it already has. For instance, let's assume a user, after
exploring a certain $WOI_A$, requests then a new $WOI_B$ that has an
overlapping with $WOI_A$. If a server maintains updated the cache model of the
content of the client cache of the user, when the $WOI_B$ is requested, the
server only sends the data-bins necessary to reconstruct the part of $WOI_B$
that is not in common with $WOI_A$. If all the data-bins that allow to
construct a certain $WOI_i$ are defined by $D(WOI_i)$, we can say that the
server sends the data-bins $D(WOI_B) - \left( D(WOI_A) \bigcap D(WOI_B) \right)$.

The JPIP protocol was designed to be independent of
the used base protocol. However, the \href{http://www.ietf.org/rfc/rfc2616.txt}
{HTTP protocol}
is mainly used as base protocol, because
of the similarity of the syntax with JPIP. Moreover,
when the communication is based on HTTP, the \href{http://en.wikipedia.org/wiki/Chunked_transfer_encoding}
{chunked transfer encoding} is commonly used. The ESA JPIP server
always uses this kind of communication.

\subsection{Data-bin partition}
\label{sec:data-bins}

Data-bins encapsulate different items of the partition
of a JPEG2000 image. This new partition of data defined
by JPIP permits to identify and gather the different parts
of an image for their transmission. In the server code,
the class \hyperlink{classjpip_1_1DataBinWriter}{jpip::DataBinWriter}
is the responsible to generate the data-bins for the
transmission.

When a client/server communication is started, it is defined
which kind of data-bin stream is going to be used. JPIP
defines two kinds of streams, JPP, oriented to precincts, and
JPT, oriented to tiles. The stream type indicates what kind
of data-bins are used for the data transmission. The most used 
stream type is JPP and is the only one supported by the ESA 
JPIP server.

Each data-bin is unequivocally identified by the image or code-stream
to what belongs, the data-bin type and its identifier. This information
is included in the header of each data-bin.

The data-bins can be segmented for theirs transmission. A certain
set of data-bins can be divided into a random number of
segments, and they can be sent following any order. Each 
segment of data-bin includes the necessary information for
its correct identification.

In Table \ref{tab:data-bins} the relation of the
different types of data-bins defined in JPIP can be observed,
showing which stream type they belongs and what information
they contain. Next the main kinds of data-bins are explained:

\begin{table}
\begin{center}
\begin{tabular}[]{{l}{l}{l}}
\textbf{Data-bin} & \textbf{Stream} & \textbf{Contained information} \\
\hline \hline 
Precinct 		& JPP 		& All the packets of a precinct. \\\hline
Precinct extended 	& JPP 		& The same content with additional content. \\\hline
Tile 			& JPT 		& All the packets and markers of a tile.\\\hline
Tile extended 		& JPT 		& The same content with additional content. \\\hline
Tile header 		& JPT 		& The markers of the header of tile. \\\hline
Header	 		& JPP/JPT 	& The markers of the header of a code-stream. \\\hline
Meta-data 		& JPP/JPT 	& Meta-data of an image. \\
\hline \hline
\end{tabular}
\caption{List of the data-bins defined by JPIP.}
\label{tab:data-bins} 
\end{center}
\end{table}

\begin{itemize}

\item \textbf{Precinct data-bin:} Contains all the packets of
a precinct for a resolution $r$, a component $c$ and a tile $t$.
The packets are ordered within a data-bin for quality layer,
in increasing order.

Every precinct data-bin is identified by an index $I$ that is obtained
by means of the following expression:

\begin{equation*}
I = t + \left( (c + (s \times N_c)) \times N_t \right)
\end{equation*}

$N_c$ y $N_t$ are the number of components and the number of tiles
of the image to which the precinct belongs, respectively. The
index of tile, $t$, as well as the index of component $c$ and
the value $s$ begin with zero. To every precinct of every tile-component,
including all the resolutions, an unique number of
sequence $s$ is assigned. To all the precincts of the lowest resolution
a number of sequence is assigned that starts with zero to the one
located in the left-top border, and that continues increasing
by column and row respectively. In the next resolutions
a correlative sequence number is assigned to the precincts in the
same way.

\item \textbf{Tile data-bin:} Contains all the packets and markers associated
to a tile. It is composed by concatenating all the tile-parts
associated to a tile, including the markers SOT, SOD and all the
rest relevant markers. The data-bins of this kind are identified by means
of an incremental index, beginning with zero for the tile
located in the top-left border.

\item \textbf{Tile header data-bin:} Contains all the markers associated to 
a certain tile. It is formed by all the markers of the headers of all the
tile-parts of a tile, without including neither the SOT nor the SOD markers.
The identifier of this data-bin has the same numeration that in the case
of the tile.

\item \textbf{Header data-bin:} Contains the main header of the code-stream,
from the SOC (included) to the first SOT marker (not included). Neither the
SOD nor the EOC marker is included either.

\item \textbf{Meta-data bins:} These data-bins appear only if
the associated image is a file of the JP2 family. The meta data-bins contain 
a set of boxes of
the image file. The standard does not define
how these data-bins must be identified nor within which boxes 
must be stored, but when a meta data-bin is identified by zero,
means that it will include all the boxes contained by an image. 

\end{itemize}

\subsection{Sessions and channels}

A request of the client to the server can be either stateful or stateless.
The stateful requests are carried out within the context of a 
communication session, which state is maintained by the server.
The stateless requests do not require any session. The use of 
sessions improve the performance of the server because, for example,
when establishing a session with a certain image file, the server
opens that file and prepares it in order to be transmitted in
data-bins, so all the requests associated to that same session
do not provoke that the server makes again that process. The stateless
requests can be considered as unique sessions that end when the
server response ends.

Under a session, using stateful requests, the client can open
multiple channels, being able to perform multiple stateful
requests associated to the same session simultaneously. This
is specially useful for applications that show simultaneously
different regions of interest of the same image. The channels
can be closed and opened in independently, without affecting
to the session. To close a session would involve to close all
the channels opened associated to it.

A set of images is associated to each session. A session
implies that the server maintains a model of the client
cache. That model will be only maintained while
the session remains active. A channel, for a certain session,
is associated to a certain image and a kind of specific
data-bin stream (JPP or JPT). The channel is identified
in a unequivocal way by means of a alphanumeric assigned
by the server, and which format is completely free. The sessions
are not identified, since the identifier of the channel
must be enough to identify the channel as well as the session
to which belongs.

Many times clients are interested to maintain in the server
the client cache model between different sessions. The server
by default does not allow this possibility, beginning a new
session with a client always with a model of cache empty, although
the client cache contains initial information. In this case JPIP
defines a set of messages that allow the client to modify the model 
that the server of its cache. Therefore, when the client starts
a new session with a server, it would send a resume of the current
content of its cache to improve the communication and to avoid
redundancy, usually by means of a POST message.

The cache models maintained by the servers not only
allow to avoid the redundancy between messages, but they
are also used in order to allow the client to perform
incremental requests and control the stream of data. Clients can 
indicate in their requests a parameter with the maximum length
of the server response. Thanks to the cache model of
the server, the client can perform successive identical requests,
but varying that parameters, so the server sends as response
increments of information. Therefore the client has certain
control and can adapt the stream of information to the available
bandwidth and delay, but always taking into account that the server
can modify whenever the related parameter. This way of communication
is indeed the most common one in the existing implementations of
JPIP.

\subsection{Messages}

The JPIP requests are formed by means of a ASCII sequence of pairs
``parameter $=$ value''. This allows that a JPIP request can be encapsulated
within a GET message of the HTTP protocol, next to the 
character '?', concatenating all the pairs with the symbol '\&'. It is
hence formed a typical HTTP request when dynamic objects are referenced
like CGI.

Some of the available parameters for a JPIP request are the followings:

\begin{itemize}

\item \textbf{``fsiz=$R_{x}$,$R_{y}$'':} Specifies the resolution
associated to the required region of interest. The server chooses
the biggest resolution of the image so its dimension $R_{x}' \times R_{y}'$
satisfies that $R_{x}' \geq R_{x}$ and $R_{y}' \geq R_{y}$. In general
this parameter includes the resolution of the user screen.

\item \textbf{``roff=$P_{x}$,$P_{y}$'':} Specifies the position of the
upper left border of the required region of interest within the
indicated resolution. If this position is not indicated, the
server assumes the value $(0,0)$.

\item \textbf{``rsiz=$S_{x}$,$S_{y}$'':} Specifies the size
of the required region of interest. The server cuts this size
in order to fit it into the real image according to the
specified resolution.

\item \textbf{``len=number of bytes'':} Client tells with
this parameter to the server the maximum number of bytes
that it can include in the response. The server takes into
account this limit, not only in the response to the
current request, but in the rest of the next client responses
within the same session.

\item \textbf{``target=image'':} This parameter identifies the image
file from which to extract the specified region of interest.
When the HTTP protocol is used, this parameter is not necessary
since the name of the image is obtained from the own URL
specified in the GET message.

\item \textbf{``cnew=protocol'':} When the client wants to open
a new channel under the same session, it uses this parameter,
indicating the protocol that must be used for this new channel. The
types of valid base protocols are ``http'' and ``http-tcp''.

\item \textbf{``cid=channel identifier'':} When the client creates
a new channel, the server sends the channel identifier, that must
be included in all the requests associated to that channel.

\item \textbf{``cclose=channel identifier'':} The client may decide
to close certain channel by means of this parameter, just specifying
the identifier of that channel.

\item \textbf{``type=stream type'':} When a new channel is created, the
client indicates what kind of data-bin stream is wanted. The main
types of data-bin streams are JPP (``jpp-stream'') and JPT
(``jpt-stream'').

\item \textbf{``model=\ldots'':} How it has been commented previously,
the client can need to tell to the server the content of its cache
in order to update the content of the cache model maintained by
the last one. For example, \texttt{model=Hm,H*,M2,P0:20}
would tell the server to include in the cache model 
the main header of the code-stream, the headers of all the tiles,
the meta data-bin number 2 and the first $20$ bytes
of the precinct $0$.
\end{itemize}

